% Unterschiedliche Klassen bieten unterschiedliche Strukturen. Article bietet
% alles was man vielleicht für ein Protokoll braucht, aber für längere Arbeiten
% hätte man vielleicht gerne Kapitel, dann lohnt sich ein Blick in "book" oder
% "report"
\documentclass[a4paper]{article} 

%\usepackage[utf8]{inputenc} % macht das Latex Dinge wie "ä, ü, ß" frisst

% vernünftige Worttrennung auf Deutsch. Für Englisch würden wir ngerman durch
% english ersetzten, alle anderen Sprachen guckt man kurz nach.
\usepackage[ngerman]{babel} 

\usepackage[separate-uncertainty=true]{siunitx} % schnieke Darstellung von Einheiten

\usepackage{booktabs}

\usepackage{amsmath} % align-Umgebungen

\usepackage{hyperref}

\usepackage[ngerman]{cleveref}

\usepackage{graphicx} % Bilder einfügen

% Kein Einrücken nach den Absätzen
\setlength{\parindent}{0pt}
\setlength{\parskip}{1ex plus 0.5ex minus 0.5ex}

% Dinge, die wir für unsere Quellensammlung brauchen
% in Texstudio muss biber eingestellt werden
\usepackage[backend=biber, style=alphabetic]{biblatex}
\addbibresource{references.bib}
\usepackage{csquotes}


\title{AP-\LaTeX kurs}
\author{Mali Heinrich \and Gregor Suchan}
\date{durchgeführt am 03.~Juli 2022}

\begin{document}

\maketitle

\tableofcontents
\section{Sprache einstellen}
Das Default-\LaTeX{} kann noch nicht alle Zeichen, die wir gerne hätten, also
ergänzen wir in unserer Präambel (also nach der \verb|\documentclass| und vor
dem \verb|\begin{document}|) folgendes: \verb|\usepackage[utf8]{inputenc}|. \#
wenn \LaTeX{} noch alt ist

Außerdem ist einer der fancy Dinge von \LaTeX, dass es selbstständig Wörter
trennt, wenn man es ihm denn sagt und damit einen wunderschönen Blocksatz
generiert.
Auch dafür nutzen wir wieder ein package und zwar wieder in der
Präambel mit dem Befehl \verb|\usepackage[ngerman]{babel}|.

\section{Struktur, Absätze, Ränder}
\LaTeX{} bietet uns unterschiedliche eingebaute Möglichkeiten die Struktur zu
kontrollieren.
Quasi das selbe wie Überschriften in Wort setzten und die Schriftgröße
einstellen, nur, dass wir hier unsere Wünsche in die Befehle schreiben und
\LaTeX{} sich um den Rest kümmert.

\subsection{Strukturelemente}
Wir hatten jetzt schon die \verb|\section{}|, die uns in diesem Fall große
Überschrif\-ten erstellt.
Wollen wir jetzt diesen Bereich unterteilen, können wir das mit dem
Befehl~\verb|\subsection{}|.
Diese basteln uns kleinere Überschriften und man sieht, dass fortlaufend
nummeriert wird.
Beide Dinge landen auch in unserem Inhaltsverzeichnis, wenn wir denn eines
erstellen.
Das geht mit \verb|\tableofcontents| und zwar an der Stelle, an der wir auch
das Verzeichnis haben wollen.

\paragraph{Paragraphen}
Wenn wir nun ein Strukturelement wollen, aber keine Nummern und das ganze soll
auch nicht ins Inhaltsverzeichnis, dann können wir einen \verb|\paragraph{}|
nutzen.
Da schreibt man aber direkt neben der Überschrift weiter.

\subsubsection*{Subsubsection mit Sternchen}
Alle nummerierten Strukturelemente können wir mit einem * versehen und
erreichen damit ein nicht-nummeriertes Element.
Dinge mit Sternchen landen aber auch nicht in unserem Inhaltsverzeichnis.
Dafür sieht das ganze quasi aus wie der Paragraph, nur dass wir nicht daneben,
sondern darunter weiterschreiben.

\subsection{Absätze und Whitespaces}
Whitespaces handhabt \LaTeX{} vielleicht etwas ungewohnt: Es ist nämlich egal,
wieviele Leerzeichen oder Tabs man        setzt.  Auch (einfache) Absätze
werden standardmäßig ignoriert.
Wenn man in \LaTeX{} einen Absatz mit einer Leerzeile im Text macht, also so

dann landet der auch in unserem Dokument. 
Die Default-Einstellung ist nun, dass Absätze eingerückt werden, wenn man
lieber einen kleinen Abstand möchte, kann man das mit
\verb|\setlength{\parindent}{0pt}| und

\verb|\setlength{\parskip}{1ex plus 0.5ex minus 0.5ex}|


\subsection{Seitenränder}
Auch hier gibt es schon Standardoptionen, sonst würden wir nichts sehen, aber
falls man daran rumspielen möchte, können wir da mit
\begin{center}
  \begin{verbatim}
    \usepackage[a4paper,
                left=3cm, right=3cm,
                top=2cm, bottom=2cm]{geometry}
  \end{verbatim}
\end{center}
Und zumindest das Papierformat sollte man irgendwo hinschreiben, sonst gibt's
ein US-Letter-Format.

\section{Formeln}
Es gibt unterschiedliche Möglichkeiten, wie wir Formeln darstellen können.
Eine Option ist einfach im Text, etwa liefert und~\verb|$a+b=c$| gerade~$a+b=$.
Die Symbole~\verb|$...$| begrenzen den Inline-Mathemodus.
Das sieht auch ganz anders aus als einfach nur~\verb|a+b=c| in den Text zu
schreiben: a+b=c.
Griechische Symbole gibt es auch:
\verb|$\beta = 2.5\gamma$| liefert uns~$\beta = 2.5\gamma$.
Aber längere Rechnungen wollen wir nicht einfach in Text machen, dafür können
wir die \verb|equation|-Umgebung benutzen, die das package \verb|amsmath|
braucht.  Das sieht dann folgendermaßen aus
\begin{equation}
  \beta = 2.5\gamma.
\end{equation}
Wenn wir nun mehrere Gleichungen haben, dann können wir die in eine
\verb|align|-Umgebung packen und machen einen Zeilenumbruch mit \verb|\\|, die
Gleichheitszeichen können wir übereinander anordnen, indem wir ein \verb|&|
davor setzen:
\begin{align}
  \beta &= 17\cdot \gamma \\
  \alpha &= \frac{5}{\delta}.
\end{align}
Das liefert uns dann aber zwei Nummern neben den Gleichungen.
Manchmal wollen wir das nicht, dann nehmen wir
\begin{align}
  \begin{split}
     \beta &= 17\cdot \gamma \label{eq:ladida}\\
    \alpha &= \frac{5}{\delta}.
  \end{split}
\end{align}
Und wenn wir gar keine Nummer wollen, dann gibt es wieder ein Sternchen:
\begin{align*}
  \beta &= 2.5\gamma \text{ und } \\
  \alpha &= \frac{5}{\delta}.
\end{align*}
Dinge, die wir vielleicht schreiben wollen sind zum Beispiel:
\begin{enumerate}
  \item \verb|\alpha \beta \gamma \rho \sigma \delta \epsilon| gibt uns in
    einer Matheumgebung, also \verb|align| oder \verb|$...$|
    $\alpha \beta \gamma \rho \sigma \delta \epsilon$.
  \item \verb|\times \otimes \oplus \cup \cap \cdot| gibt $\times \otimes
    \oplus \cup \cap \cdot$.
  \item \verb|< > \subset \supset \subseteq \supseteq \leq \geq| gibt 
    \begin{equation}
      < > \subset \supset \subseteq \supseteq \leq \geq.
    \end{equation}
  \item \verb|\int \oint \sum \prod| gibt $\int \oint \sum \prod$.
  \item \verb|F_{\text{rück}}| gibt $F_{\text{rück}}$, der Unterstrich gibt uns
    also ein Subskript und wir gehen in eine Text-Umgebung, weil der Text sonst
    kursiv wird ($Text\neq \text{Text} $).
  \item analog: \verb|F_{g} G_{\epsilon} h_{\epsilon\delta}| gibt $F_{g}
    G_{\epsilon} h_{\epsilon\delta}$.
  % \item Wenn wir aus sonst welchen Gründen mehr Platz haben wollen: \verb|f g~h~~i~~\ j| gibt $f g~h~~i~~~j$
  \item Normalerweise werden Whitespaces im Mathemodus vollständig ignoriert.
    Stattdessen darf man diese dann manuell setzen:
    \begin{center}
      \begin{tabular}{ccc}
        \verb|a\!b|      & gibt uns & $a\!b$, \\
        \verb|ab|        & gibt uns & $ab$, \\
        \verb|a\,b|      & gibt uns & $a\,b$, \\
        \verb|a\:b|      & gibt uns & $a\:b$, \\
        \verb|a\ b|      & gibt uns & $a\ b$, \\
        \verb|a\quad b|  & gibt uns & $a\quad b$, \\
        \verb|a\qquad b| & gibt uns & $a\qquad b$. \\
      \end{tabular}
    \end{center}
  \item Integrale \verb|\int_{0}^{\infty} f(x)dx| macht $\int_{0}^{\infty}
    f(x)dx$.
  \item Je nach Stil ist das d in Integralen aufrecht und mit einem halben
    Abstand \verb|\int_0^\infty f(x){\,}\mathrm{d}x| macht $\int_0^\infty
    f(x)\,\text{d}x$.
  \item \verb|\sum_{i=1}^{n} i^2 = \frac{n(n+1)(2n+1)}{6}| gibt 
    \begin{equation}
      \sum_{i=1}^n i^2 = \frac{n(n+1)(2n+1)}{6}.
    \end{equation}
  \item \verb|\frac{\partial^2 f}{\partial x^2} f'(c)| gibt $\frac{\partial^2
    f}{\partial x^2}f'(c)$.
\end{enumerate}

\section{Bilder einfügen}
Wir brauchen auch zum Einfügen von Bildern ein package und schreiben
dafür einfach~\verb|\usepackage{graphicx}| mit in die Präambel.

Und jetzt können wir Bilder einfügen mit
\begin{center}
  \begin{verbatim}
  \begin{figure}[htbp]
    \centering
    \includegraphics[width=0.6\textwidth]{cat.png}
    \caption{Bild von einer Katze}%
    \label{fig:cat}
  \end{figure}
  \end{verbatim}
\end{center}

\begin{figure}[bp]
  \centering
  \includegraphics[width=0.7\textwidth]{cat.png}
  \caption{Bild von einer Katze}%
  \label{fig:cat}
\end{figure}

und dem ganzen eine schöne Bildunterschrift geben und ein Label, auf das wir
dann später verweisen können.

Figures sind Beispiele von Floating-Umgebungen.
Diese werden automatisch nicht relativ zum umgebenden Text gesetzt,
sondern~\LaTeX{} packt diese einfach an den besten Ort -- oder zumindest an den
Ort, den~\LaTeX{} für den besten erachtet.
Mit den Argumenten htbp (here, top, bottom, (next) page) in den eckigen
Klammern kann man versuchen das Bild durch die Gegend zu schieben, das klappt
mehr oder weniger gut.

\section{Referenzen oder Verweise}
\label{sec:referenzen_verweise}
Wenn wir an irgendeinem späteren Punkt den Leser mal wieder daran erinnern
wollen, dass wir schon mal was gesagt haben oder einfach nur auf ein Bild
hinweisen wollen, hat \LaTeX{} da Dinge für uns parat: zum Beispiel in
Form von \verb|\usepackage[ngerman]{cleveref}|.
So können wir an unsere Gleichung erinnern mit \verb|\cref{eq:ladida}|, das
gibt dann \cref{eq:ladida}.
Wenn wir jetzt noch  \verb|\usepackage{hyperref}| vor das cleveref-package
setzen, dann ist der Verweis nice and clicky.
Verweisen können wir auf quasi alles auf unser Bild \cref{fig:cat} oder auf
diesen \cref{sec:referenzen_verweise}.

\section{Tabellen}
Einen nicht ganz unerheblichen Teil der Zeit verbringt man im AP damit, Werte
schön in Tabellen darzustellen.
Das ist in \LaTeX{} zwar nicht angenehmer als anderswo (und zwar gar nicht),
aber liefert uns dafür auch ein paar angenehme Dinge wie Tabellennummerierung,
Tabellenunterschrift und die Möglichkeit auf Tabellen zu verweisen.
Das Einfügen von Tabellen funktioniert nun ganz ähnlich wie das von Bildern.

\begin{table}[htbp]
  \centering
  \begin{tabular}{c|c|c|c}
    Wert 1 & Wert 2 & Wert 3 & Wert 4 \\ \hline
    1      & 35     & 12     & 15 \\
    32     & 12     & 33.5   & 1 \\
  \end{tabular}
  \caption{Unsere erste Tabelle}%
  \label{tab:erste_tabelle}
\end{table}

Wenn wir nun einen Blick in \cref{tab:erste_tabelle} werfen, ist das ja schon
fast zufriedenstellend.
Wenn wir nun Werte mit Einheiten haben, können wir die Einheiten oben in die
Überschrift packen (zu Empfehlen) oder auch hinter die Zahlen schreiben (wenn
es aus irgendeinem Grund unterschiedliche Einheiten sein sollten).
Dazu verwenden wir das SI-package, was wir aber erst mal importieren müssen.
Also schreiben wir \verb|\usepackage{siunitx}| in die Präambel und können
unsere Tabelle noch mal schreiben.

\begin{table}[htbp]
  \centering
  \begin{tabular}{c|c|c|c}
    Wert 1 & Wert 2 & Wert 3 & Wert 4 \\ \hline
    \SI{1.0}{\meter} & \SI{35e-3}{\joule\per\second} & \num{12} & \SI{15.03(4)}{\pascal} \\
    \num{32} & \num{12} & \num{33.5} & \num{1}
  \end{tabular}
  \caption{Unsere zweite Tabelle}
  \label{tab:zweite_tabelle}
\end{table}

Wenn wir mit \verb|\SI{}{}| und \verb|\num{}| arbeiten, haben wir auch den
charmanten Vorteil, dass wir nicht drüber nachdenken müssen, ob wir unsere
Kommazahlen mit Kommata oder Punkten trennen, dass können wir einfach nach Lust
und Laune oben in der Präambel ändern.

\begin{table}[htbp]
  \centering
  \begin{tabular}{
      S[table-format=3.1(1)]
      S[table-format=2.3(3)]
      S[table-format=2.1(1)]
      S[table-format=2.0]
    }
    {Wert 1 $[\text{mm}]$} & {Wert 2 $[\text{mm}]$} & {Wert 3 $[\text{mm}]$} & {Wert 4 $[\text{mm}]$} \\ \hline
    122.7(1)               & 35.099(123)            & 12.0(5)                & 15                     \\
    32                     & 12(3)                  & 33.5(1)                & 1                      \\
  \end{tabular}
  \caption{Unsere dritte Tabelle}%
  \label{tab:dritte_tabelle}
\end{table}

Mit ein bisschen Kampf (aber das kann man ja beim nächsten Protokoll einfach
kopieren und einfügen) können wir nun auch unsere Werte am Komma ausrichten
und hübsch formatierte Unsicherheiten dazubasteln.
Da das ganze immer noch die Dinge vom SI-package frisst, haben wir auch
trotzdem kein Problem mit dem Komma-oder-Punkt-Ding.

\section{Referenzenmanagement}
Wenn man wissenschaftlich arbeitet, tut man dies selten originell.
Oft verweist man auf Arbeiten anderer Leute, die vor einem kamen.
Das macht man wie folgt: Oben in die Präambel schreiben wir wieder nette Dinge
rein und zwar~\verb|\usepackage[backend=biber, style=alphabetic]{biblatex}|
    um unser package zu laden und wir sagen ihm auch wo wir unsere Quellen
    finden. Hier haben wir die in einem extra Text-file gespeichert und zwar
    heißt der \emph{references.bib}. Das sieht dann so aus:
\verb|\addbibresource{references.bib}|. In diesen Textfile schreiben wir unsere
Quellen in einer besonderen, aber wichtigen Syntax. Es gibt die unangenehmsten
Fehlermeldungen, wenn man das versemmelt.


\begin{center}
  \begin{verbatim}
@Book{                citekey,
   title        = {Titel vom Buch},
   author       = {Nachname, Vorname and Zweitautornachn., Zvorname },
   year         = {1986},
   publisher    = {Verlag},
}
  \end{verbatim}
\end{center}
Im Text macht man dann einfach~\verb|\cite{citekey}|~\cite{Knuth1986}.
Referenzen sollte man immer mal zu Beginn eines Schreibprozesses einbauen, dann
weiß man gleich obs klappt und es haben schon viele vor euch versucht nur noch
mal kurz die Referenzen einzufügen und hatten dann ne ziemlich unangenehme
Nacht.

Wenn wir dann unsere Quellen alle zusammenhaben, dann können wir den Spaß mit 
\verb|\printbibliography| dann drucken. Und wir auch unser Inhaltsverzeichnis
erscheint das ganze dann an genau der Stelle, wo wir auch den Befehl dazu
schreiben.

\printbibliography
\end{document}
